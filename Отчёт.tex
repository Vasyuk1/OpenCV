%% -*- coding: utf-8 -*-
\documentclass[14pt,a4paper]{scrartcl} 
\usepackage[utf8]{inputenc}
\usepackage[english,russian]{babel}
\usepackage{indentfirst}
\usepackage{misccorr}
\usepackage{graphicx}
\usepackage{amsmath}
\usepackage{listings}
\begin{document}
	\begin{titlepage}
		\begin{center}
			\large
			МИНИСТЕРСТВО НАУКИ И ВЫСШЕГО ОБРАЗОВАНИЯ РОССИЙСКОЙ ФЕДЕРАЦИИ
			
			Федеральное государственное бюджетное образовательное учреждение высшего образования
			
			\textbf{АДЫГЕЙСКИЙ ГОСУДАРСТВЕННЫЙ УНИВЕРСИТЕТ}
			\vspace{0.25cm}
			
			Инженерно-физический факультет
			
			Кафедра автоматизированных систем обработки информации и управления
			\vfill

			\vfill
			
			\textsc{Отчет по практике}\\[5mm]
			
			{\LARGE Создать программу для обработки изображений, которая будет
выполнять следующие функции: изменение яркости, контрастности и
насыщенности цветов}
			\bigskip
			
			2 курс, группа 2ИВТ2
		\end{center}
		\vfill
		
		\newlength{\ML}
		\settowidth{\ML}{«\underline{\hspace{0.7cm}}» \underline{\hspace{2cm}}}
		\hfill\begin{minipage}{0.5\textwidth}
			Выполнил:\\
			\underline{\hspace{\ML}} С.\,С.~Васюк\\
			«\underline{\hspace{0.7cm}}» \underline{\hspace{2cm}} 2024 г.
		\end{minipage}%
		\bigskip
		
		\hfill\begin{minipage}{0.5\textwidth}
			Руководитель:\\
			\underline{\hspace{\ML}} С.\,В.~Теплоухов\\
			«\underline{\hspace{0.7cm}}» \underline{\hspace{2cm}} 2024 г.
		\end{minipage}%
		\vfill
		
		\begin{center}
			Майкоп, 2024 г.
		\end{center}
	\end{titlepage}
\tableofcontents
\newpage
\section{Введение}
Для разработки алгоритма использовалась среда разработки Visual Studio (в дальнейшем VS) на языке прогроммирования С++, а также специальная библиотека OpenCV для работы с изображениями и видео. При открытии VS запускается следующее окно (рисунок 1):
\begin{figure}[h!]
    \centering
    \includegraphics [width=0.9\textwidth]{pic1}\\
    \caption{Окно запуска проектов Visual Studio}
    \label{fig:pic1}
\end{figure}

\newpage

Для отрытия проекта необходимо кликнуть на нужное наименование после чего проект благополучно запуститься и можно начинать редактирование и компиляцию кода (рисунок 2): 

\begin{figure}[h!]
    \centering
    \includegraphics [width=0.9\textwidth]{pic2}\\
    \caption{Окно проекта Visual Studio}
    \label{fig:pic2}
\end{figure}

\newpage
\section{Работа программы}
На рисунке 2 можно увидеть алгоритм решения задачи на редактирование изображения по трём параметрам: яркость, контрастность и насыщенность. После запуска программы необходимо ввести путь к изображению, а после указать значения параметров. Результат его работы отображён на рисунке 3:

\begin{figure}[h!]
    \centering
    \includegraphics [width=0.9\textwidth]{pic3}\\
    \caption{Окно проекта Visual Studio}
    \label{fig:pic3}
\end{figure}
\newpage
\section{Код программы}


\begin{verbatim}
#include <opencv2/opencv.hpp>
#include <iostream>

using namespace cv;
using namespace std;

int main(int argc, char** argv)
{
    setlocale(LC_ALL, "Russian");
    string picDerectory;
    cout << "Введите путь к изображению 
    (например: F:\\manga\\01-02-02bratva001.jpeg):" << endl;
    cin >> picDerectory;
    Mat image = imread(picDerectory);
    resize(image, image, Size(500, 500));
    
    if (image.empty())
    {
        cout << "Could not open or find the image" << endl;
        cin.get();
        return -1;
    }

    int brighness = 0;
    int contrast = 0;
    int saturation = 0;
    int count;
    cout << "Введите параметр яркости (от -100 до 100): " << endl;
    cin >> brighness;
    cout << "Введите параметр контрастности (от -10 до 10): " << endl;
    cin >> contrast;
    cout << "Введите параметр цветовой насыщенности (от -10 до 10): " 
    << endl;
    cin >> saturation;

    Mat imageBrighnessHigh50;
    Mat convertedHSV;
    image.convertTo(imageBrighnessHigh50, -1, contrast, brighness); 
        
    cvtColor(imageBrighnessHigh50, convertedHSV, COLOR_BGR2HSV);
    Mat HSVChannels[3];
    split(convertedHSV, HSVChannels);
    HSVChannels[1] *= saturation;
    merge(HSVChannels, 3, imageBrighnessHigh50);
    cvtColor(imageBrighnessHigh50, imageBrighnessHigh50, COLOR_HSV2BGR);

    String windowNameOriginalImage = "Оригинальное изображение";
    String windowNameBrightnessHigh50 = "Отредактированное изображение";

    namedWindow(windowNameOriginalImage, WINDOW_NORMAL);
    namedWindow(windowNameBrightnessHigh50, WINDOW_NORMAL);

    imshow(windowNameOriginalImage, image);
    imshow(windowNameBrightnessHigh50, imageBrighnessHigh50);
    waitKey(0); 
    destroyAllWindows();
    return 0;
}
\end{verbatim}

\newpage
\section{Список используемой литературы}

\begin{enumerate}

\item ОБРАБОТКА ИЗОБРАЖЕНИЙ В C++ С ПОМОЩЬЮ БИБЛИОТЕКИ OpenCV //
Universum: технические науки : электрон. научн. журн. Самандаров И.Р. [и др.]. 2023. 5(110). 
URL: https://7universum.com/ru/tech/
archive/item/15484
\item Официальная документация Open Source Computer Vision. 

URL: https://docs.opencv.org/4.9.0/examples.html
\end{enumerate}

\end{document}
